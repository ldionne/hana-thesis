\section{Les transformations naturelles}
Nous venons de voir la notion de foncteur, qui peuvent être vus comme
des transformations entre les catégories. Pour continuer dans la même
direction, nous allons maintenant considérer ce que sont les transformations
entre foncteurs. Ces transformations portent le nom de transformations
naturelles et jouent un rôle fondamental en théorie des catégories.

Mais que veut-on dire exactement par transformation entre foncteurs?
Étant donné deux foncteurs $F : \C \to \D$ et $G : \C \to \D$, on aimerait
avoir une transformation ...

\todo{Intuition. Comment arriver à la définition formelle graduellement?}

\begin{définition}[Transformation naturelle]
    Soit $\C$ et $\D$ deux catégories. Une transformation naturelle entre
    deux foncteurs $F : \C \to \D$ et $G : \C \to \D$ est une application
    $\eta : \ob(\C) \to \hom(\D)$ qui à chaque objet $X \in \ob(\C)$ associe
    une flèche $\eta(X) : F(X) \to G(X) \in \hom(\D)$ telle que pour tout
    objet $Y \in \ob(\C)$ et toute flèche $f : X \to Y \in \hom(\C)$, on a
    \[
        \eta(Y) \circ F(f) = G(f) \circ \eta(X)
    \]

    Pour dire que $\eta$ est une transformation naturelle entre $F$ et $G$,
    on notera parfois $\eta : F \to G$. De plus, on appellera $\eta(X)$ la
    composante de $\eta$ en $X$.
\end{définition}

On notera que dans la littérature, $\eta$ est aussi souvent présentée comme
une famille de flèches qui à chaque objet $X \in \ob(\C)$ associe une flèche
$\eta_X : F(X) \to G(X) \in \hom(\D)$. Les deux présentations sont
évidemment équivalentes.

Assurons-nous maintenant que l'égalité $\eta(Y) \circ F(f) = G(f) \circ \eta(X)$
fait bien du sens. Du côté gauche, on a que $\eta(Y) : F(Y) \to G(Y)$ est une
flèche de $\D$ et que $F(f) : F(X) \to F(Y)$ est aussi une flèche de $\D$.
Ainsi, la composition de ces deux flèches fait du sens et nous donne une flèche
$\eta(Y) \circ F(f) : F(X) \to G(Y)$ de $\D$. Du côté droit, on a que
$G(f) : G(X) \to G(Y)$ est une flèche de $\D$ et que $\eta(X) : F(X) \to G(X)$
est aussi une flèche de $\D$. Ainsi, la composition de ces deux flèches fait
du sens et nous donne une flèche $G(f) \circ \eta(X) : F(X) \to G(Y)$ de $\D$.
Si elle n'est pas très intuitive, l'égalité de ces deux expressions fait donc
au moins du sens.

Or, il existe une manière visuelle de se représenter cette équation qui nous
aidera à en tirer un peu d'intuition. D'abord, on représente les catégories
$\C$ et $\D$ ainsi que l'action des foncteurs $F : \C \to \D$ et $G : \C \to \D$
sur deux objets quelconques $X,Y \in \ob(\C)$ et une flèche $f : X \to Y$.

\todo{Compléter ce diagramme.}
\begin{tikzpicture}
    \draw (2,2) ellipse (1.5cm and 2cm); % Catégorie C
    \draw (2,1) node {.}; % X
    \draw (2,3) node {.}; % Y

    \draw (8,2) ellipse (1.5cm and 2cm); % Catégorie D
    \draw (7,1) node {.}; % F(X)
    \draw (9,1) node {.}; % G(X)

    \draw (7,3) node {.}; % F(Y)
    \draw (9,3) node {.}; % G(Y)
\end{tikzpicture}

Ajoutons maintenant la représentation de $\eta$ appliquée à ces objets. On
obtient

\todo{Compléter ce diagramme.}
\begin{tikzpicture}
    \draw (2,2) ellipse (1.5cm and 2cm); % Catégorie C
    \draw (2,1) node {.}; % X
    \draw (2,3) node {.}; % Y

    \draw (8,2) ellipse (1.5cm and 2cm); % Catégorie D
    \draw (7,1) node {.}; % F(X)
    \draw (9,1) node {.}; % G(X)

    \draw (7,3) node {.}; % F(Y)
    \draw (9,3) node {.}; % G(Y)
\end{tikzpicture}

L'équation $\eta(Y) \circ F(f) = G(f) \circ \eta(X)$ peut se lire comme
``$\eta(Y)$ après $F(f)$ doit être égal à $G(f)$ après $\eta(X)$'', ce
qui veut simplement dire que les chemins rouges et bleus doivent en fait
donner le même résultat, et ce peut importe les $X$, $Y$ et $f$ considérés.
Ainsi, l'équation présentée plus haut peut se réécrire sous une forme plus
visuelle en affirmant que le diagramme suivant commute:

\[
\begin{tikzcd}
    F(X) \arrow[d, "\eta(X)"'] \arrow[r, "F(f)"] & F(Y) \arrow[d, "\eta(Y)"] \\
    G(X) \arrow[r, "G(f)"] & G(Y)
\end{tikzcd}
\]

Pour aller de l'avant dans notre tentative de catégorifier autant que possible,
nous remarquerons qu'il est possible de voir les transformations naturelles
comme les flèches d'une catégorie bien spéciale. Il s'agit de la catégorie
$Fun(\C, \D)$ des foncteurs entre deux catégories $\C$ et $\D$, ce qui nous
mène à la définition suivante.

\begin{définition}[Catégorie de foncteurs]
    Soit $\C$ et $\D$ deux catégories. On appelle $Fun(\C, \D)$ la catégorie
    des foncteurs entre $\C$ et $\D$, dont les objets sont les foncteurs
    (covariants) entre $\C$ et $\D$. De plus, pour deux objets $F$ et $G$ de
    la catégorie, l'ensemble des flèches $\hom(F, G)$ est l'ensemble des
    transformations naturelles entre $F$ et $G$. Finalement, la loi de
    composition de la catégorie est la composition usuelle de foncteurs.
\end{définition}


\subsection{Les transformations naturelles dans Hana}
