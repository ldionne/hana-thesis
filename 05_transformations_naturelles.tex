\section{Les transformations naturelles}
Nous venons de voir la notion de foncteur, qui peuvent être vus comme
des transformations entre les catégories. Pour continuer dans la même
direction, nous allons maintenant considérer ce que sont les transformations
entre foncteurs. Ces transformations portent le nom de transformations
naturelles et jouent un rôle fondamental en théorie des catégories.

Mais que veut-on dire exactement par transformation entre foncteurs?
Étant donné deux foncteurs $F : \C \to \D$ et $G : \C \to \D$, on aimerait
avoir une transformation ...

\todo{Intuition. Comment arriver à la définition formelle graduellement?}

\begin{définition}[Transformation naturelle]
    Soit $\C$ et $\D$ deux catégories. Une transformation naturelle
    $\eta : F \to G$ entre deux foncteurs $F : \C \to \D$ et $G : \C \to \D$
    est une famille de flèches qui à chaque objet $X \in \ob(\C)$ associe une
    flèche $\eta_X : F(X) \to G(X) \in \hom(\D)$ telle que pour tout objet
    $Y \in \ob(\C)$ et toute flèche $f : X \to Y \in \hom(\C)$, on a
    \[
        \eta_Y \circ F(f) = G(f) \circ \eta_X
    \]
\end{définition}

Le fait que $\eta$ soit une famille de flèches plutôt qu'une application
peut sembler un peu déconcertant. Il suffit alors de se rappeler qu'une
famille indexée n'est en fait qu'une fonction sous le couvert. Ainsi, on
aurait aussi pu voir $\eta$ comme une application $\eta : \ob(\C) \to \hom(\D)$
qui à chaque objet $X \in \ob(\C)$ associe une flèche
$\eta(X) : F(X) \to G(X) \in \hom(\D)$. On préfère ici la notation indexée
parce qu'elle est plus commune dans la littérature et qu'elle nous évite
certaines difficultés de notation un peu plus loin.

Assurons-nous maintenant que l'égalité $\eta_Y \circ F(f) = G(f) \circ \eta_X$
fait bien du sens. Du côté gauche, on a que $\eta_Y : F(Y) \to G(Y)$ est une
flèche de $\D$ et que $F(f) : F(X) \to F(Y)$ est aussi une flèche de $\D$.
Ainsi, la composition de ces deux flèches fait du sens et nous donne une flèche
$\eta_Y \circ F(f) : F(X) \to G(Y)$ de $\D$. Du côté droit, on a que
$G(f) : G(X) \to G(Y)$ est une flèche de $\D$ et que $\eta_X : F(X) \to G(X)$
est aussi une flèche de $\D$. Ainsi, la composition de ces deux flèches fait
du sens et nous donne une flèche $G(f) \circ \eta_X : F(X) \to G(Y)$ de $\D$.
Si elle n'est pas très intuitive, l'égalité de ces deux expressions fait donc
au moins du sens.

Or, il existe une manière visuelle de se représenter cette équation qui nous
aidera à en tirer un peu d'intuition. D'abord, on représente les catégories
$\C$ et $\D$ ainsi que l'action des foncteurs $F : \C \to \D$ et $G : \C \to \D$
sur deux objets quelconques $X,Y \in \ob(\C)$ et une flèche $f : X \to Y$.

\todo{Compléter ce diagramme.}
\begin{tikzpicture}
    \draw (2,2) ellipse (1.5cm and 2cm); % Catégorie C
    \draw (2,1) node {.}; % X
    \draw (2,3) node {.}; % Y

    \draw (8,2) ellipse (1.5cm and 2cm); % Catégorie D
    \draw (7,1) node {.}; % F(X)
    \draw (9,1) node {.}; % G(X)

    \draw (7,3) node {.}; % F(Y)
    \draw (9,3) node {.}; % G(Y)
\end{tikzpicture}

Ajoutons maintenant la représentation de $\eta$ appliquée à ces objets. On
obtient

\todo{Compléter ce diagramme.}
\begin{tikzpicture}
    \draw (2,2) ellipse (1.5cm and 2cm); % Catégorie C
    \draw (2,1) node {.}; % X
    \draw (2,3) node {.}; % Y

    \draw (8,2) ellipse (1.5cm and 2cm); % Catégorie D
    \draw (7,1) node {.}; % F(X)
    \draw (9,1) node {.}; % G(X)

    \draw (7,3) node {.}; % F(Y)
    \draw (9,3) node {.}; % G(Y)
\end{tikzpicture}

L'équation $\eta_Y \circ F(f) = G(f) \circ \eta_X$ peut se lire comme
``$\eta_Y$ après $F(f)$ doit être égal à $G(f)$ après $\eta_X$'', ce
qui veut simplement dire que les chemins rouges et bleus doivent en fait
donner le même résultat, et ce peut importe les $X$, $Y$ et $f$ considérés.
Ainsi, l'équation présentée plus haut peut se réécrire sous une forme plus
visuelle en affirmant que le diagramme suivant commute:
\[
\begin{tikzcd}
    F(X) \arrow[d, "\eta_X"'] \arrow[r, "F(f)"] & F(Y) \arrow[d, "\eta_Y"] \\
    G(X) \arrow[r, "G(f)"] & G(Y)
\end{tikzcd}
\]

\subsection{Les transformations naturelles dans Hana}
Dans Hana, un foncteur est un type généralisé paramétré et une flèche
$f : X \to Y$ est une fonction d'un type généralisé $X$ vers un type
généralisé $Y$. Ainsi, étant donné deux types généralisés paramétrés $F$
et $G$ qui sont des foncteurs, une transformation naturelle sera une
famille de fonctions qui à chaque type généralisé $X$ associe une fonction
$\eta_X : F(X) \to G(X)$. De plus, la loi énoncée plus haut se traduit comme
suit dans le langage de Hana: Pour tout type généralisé $Y$ et toute fonction
$f : X \to Y$, on doit avoir
\begin{cpp}
    to<G(Y)> . transform(-, f) == transform(-, f) . to<G(X)>
\end{cpp}

Par exemple, qu'arriverait-il si on choisissait $F = \icpp{Maybe}$ et $G = \icpp{Tuple}$?
On obtiendrait alors une transformation naturelle \icpp{to<Tuple(X)>} de
type $Maybe(X) \to Tuple(X)$ qui devrait faire en sorte que pour tout type
généralisé $Y$ et toute fonction $f : X \to Y$, on ait
\begin{cpp}
    to<Tuple(Y)> . transform(-, f) == transform(-, f) . to<Tuple(X)>
\end{cpp}
