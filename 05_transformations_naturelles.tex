\section{Les transformations naturelles}
Nous venons de voir la notion de foncteur, qui peuvent être vus comme
des transformations entre les catégories. Pour continuer dans la même
direction, nous allons maintenant considérer ce que sont les transformations
entre foncteurs. Ces transformations portent le nom de transformations
naturelles et jouent un rôle fondamental en théorie des catégories.

\todo{
    Qu'est-ce qu'on veut dire par morphisme entre foncteurs?
    Quelles propriétés est-ce qu'on veut? Introduire les transformations
    naturelles de manière graduelle en faisant ce travail là. Ensuite
    on peut donner la définition formelle. Diagrammes pour montrer
    l'intuition.
}

Pour aller de l'avant dans notre tentative de catégorifier autant que possible,
nous remarquerons qu'il est possible de voir les transformations naturelles
comme les flèches d'une catégorie bien spéciale. Il s'agit de la catégorie
$Fun(\C, \D)$ des foncteurs entre deux catégories $\C$ et $\D$, ce qui nous
mène à la définition suivante.

\begin{définition}[Catégorie de foncteurs]
    Soit $\C$ et $\D$ deux catégories. On appelle $Fun(\C, \D)$ la catégorie
    des foncteurs entre $\C$ et $\D$, dont les objets sont les foncteurs
    (covariants) entre $\C$ et $\D$. De plus, pour deux objets $F$ et $G$ de
    la catégorie, l'ensemble des flèches $\hom(F, G)$ est l'ensemble des
    transformations naturelles entre $F$ et $G$, qui, on se rappelle, sont
    des foncteurs. Finalement, la loi de composition de la catégorie est
    la composition usuelle de foncteurs.
\end{définition}


\subsection{Les transformations naturelles dans Hana}
