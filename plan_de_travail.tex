\documentclass{article}
\usepackage[utf8]{inputenc}

\title{Plan de travail}
\author{Louis Dionne}
\date{23 Janvier 2015}

\begin{document}

\maketitle


\section{Objectif du projet}
Montrer comment la théorie des catégories peut être utilisée pour formaliser
certains aspects de la métaprogrammation statique en C++.

À noter que j'ai décidé d'exclure la partie traitant d'algèbre abstraite. En
effet, celle-ci m'intéressait moins et tombait plus sous le sens. De plus,
j'estime que parler de la théorie des catégories devrait être une tâche
suffisante.

\section{Démarche}
Consulter les excellentes ressources disponibles sur internet faisant des
liens entre le Haskell et la théorie des catégories. Appliquer la théorie
à la métaprogrammation en C++ au lieu du Haskell; cette partie n'a jamais
été faite auparavant et j'irai donc un peu comme on fait de la recherche;
par tâtonnement dirigé et en me basant sur ce qui existe pour le Haskell.

\section{Organisation du travail}
À noter que les nombres de pages sont à titre indicatif seulement et seront
certainement amenés à changer. Ils représentent pour moi un indicateur du
niveau de détails dans lequel je souhaite entrer.

\begin{itemize}
    \item Introduction: (environ 1 page)
        \begin{itemize}
            \item La programmation fonctionnelle
                \begin{itemize}
                    \item Lien avec les mathématiques
                    \item Haskell et la théorie des catégories
                \end{itemize}
            \item Le C++
                \begin{itemize}
                    \item L'apparition des templates
                    \item La Turing-complétude accidentelle
                    \item La métaprogrammation
                    \item Lien avec la programmation fonctionnelle
                \end{itemize}
        \end{itemize}

    \item Introduction à la métaprogrammation en C++ (environ 6 pages)
        \begin{itemize}
            \item Les templates (environ 1 page)
            \item La métaprogrammation classique style MPL (environ 2 pages)
                \begin{itemize}
                    \item Représentation des valeurs par des types
                    \item Séquences de type
                \end{itemize}
            \item La métaprogrammation dans Hana (environ 2 pages)
                \begin{itemize}
                    \item Utilisation de la déduction de type
                    \item Utilisation de singletons pour les calculs sur des types
                    \item Séquences hétérogènes
                \end{itemize}
            \item Les applications (environ 1 page)
            \item Lien avec la programmation fonctionnelle (avec le lambda calcul?)
        \end{itemize}

    \item Introduction aux catégories (environ 5 pages)
    \begin{itemize}
        \item Définition formelle (environ 1 page)
        \item Quelques exemples de base (environ 2 pages)
            \begin{itemize}
                \item Catégories presque triviales
                \item Set
                \item Grp
            \end{itemize}
        \item La catégorie Hask (environ 1 page)
        \item La catégorie Hana (environ 1 page)
    \end{itemize}

    \item La théorie des catégories dans Hana (environ 15 pages total)
    \begin{itemize}
        \item Foncteurs (environ 5 pages)
        \begin{itemize}
            \item Définition formelle
            \item Définition dans Hana
            \item Intuition
            \item Réalisation dans Hana: Maybe, Either, Tuple
        \end{itemize}

        \item Transformations naturelles (environ 2 pages)
        \begin{itemize}
            \item Définition formelle
            \item Exemples de transformations naturelles dans Hana
        \end{itemize}

        \item Monades (environ 5 pages)
        \begin{itemize}
            \item Définition formelle
            \item Définition dans Hana
            \item Intuition
            \item Réalisation dans Hana + preuves: Maybe, Either, Tuple
        \end{itemize}

        \item Produits (environ 3 pages)
        \begin{itemize}
            \item Définition formelle
            \item Définition dans Hana
            \item Réalisation dans Hana + preuves: Pair
        \end{itemize}

        \item Bonus (si temps et si j'y arrive)
        \begin{itemize}
            \item Définition de la catégorie MPL
            \item Formalisation du foncteur $Type : MPL \to Hana$:
            La fidélité du foncteur nous donnerait une preuve que tout ce
            qui peut être fait dans MPL peut être fait dans Hana. En effet,
            si les flèches de MPL sont des métafonctions, je pense que la
            fidélité du foncteur nous donnerait qu'il est possible d'exprimer
            toutes les métafonctions de MPL (et peut-être plus) dans Hana. Il
            me reste du défrichage à faire pour me rendre là.
        \end{itemize}
    \end{itemize}

    \item Conclusion (revue des contributions) (environ 1 page)

\end{itemize}

\section{Échéancier}
\begin{itemize}
    \item 23 janvier: Remise du plan de travail
    \item 24 janvier - 18 février:
    \begin{itemize}
        \item Squelette du travail au complet
        \item Version presque finale de l'introduction aux catégories et de la
        théorie des catégories dans Hana (sauf le bonus). Je souhaite avoir
        une version presque finale de ces parties très rapidement parce que
        c'est là que je prévois de rencontrer le plus de difficultés.
    \end{itemize}
    \item 19 février: Rapport d'étape
    \item 20 février - 23 avril:
    \begin{itemize}
        \item Rédiger l'introduction et la conclusion
        \item Compléter l'introduction à la métaprogrammation en C++.
        \item Commencer à monter la présentation orale
        \item Rédiger la partie bonus si le temps le permet
    \end{itemize}
    \item 24 avril: Remise du document final
    \item 25 avril - 27 avril: Finaliser le travail sur la présentation orale
    \item 28 avril: Présentation orale
\end{itemize}

\end{document}
