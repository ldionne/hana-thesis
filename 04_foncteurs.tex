\section{Les foncteurs}
Nous avons introduit la notion de catégorie à l'aide d'exemples provenant de
l'algèbre abstraite, de l'analyse puis de l'informatique. L'algorithme pour
\textit{catégoriser} une théorie était toujours un peu le même: isoler les
objets d'étude de la théorie, isoler les propriétés importantes qui les
définissent puis considérer l'ensemble des transformations qui, de manière
informelle, respectent ces propriétés. Or, la théorie des catégorie est
elle-même une théorie qui étudie certains objets appelés ``catégories''.
Il serait donc naturel de se demander s'il est possible de définir des
transformations entre les catégories et, si c'est le cas, quelles propriétés
doivent posséder ces transformations pour qu'on obtienne quelque chose
d'intéressant. Comme on s'apprête à le voir, ce questionnement et sa
réponse sont à la source de la notion de \textit{foncteur}, qui n'est
en effet rien d'autre qu'une transformation entre catégories.

D'abord, une catégorie $\C$ possède non seulement des objets, mais aussi
des flèches entre ces objets. Ainsi, peu importe la forme que prendra notre
transformation inter-catégories, elle devra spécifier où elle envoie les
objets, mais aussi les flèches. Considérons donc une application $F : \C \to \D$
d'une catégorie $\C$ vers une catégorie $\D$. On devra avoir que pour chaque
objet $A \in \ob(\C)$, $F$ envoie $A$ vers un objet de $\D$, disons
$F(A) \in \ob(\D)$. De manière similaire, pour chaque flèche
$f : A \to B \in \hom(\C)$, $F$ devra envoyer $f$ sur une flèche
de $\D$, disons $F(f) \in \hom(\D)$.

Il est intéressant de remarquer que l'on a pas spécifié la source ni la
destination de la flèche $F(f)$. En fait, pour une flèche $f : A \to B$,
seuls deux choix s'offrent à nous: soit $F(f) : F(A) \to F(B)$, soit
$F(f) : F(B) \to F(A)$. Ces deux choix sont valables. Dans le premier cas,
on dira que c'est un foncteur \textit{covariant}, ou juste un foncteur, et
dans l'autre cas on dira que c'est un foncteur \textit{contravariant}. Dans
le cadre actuel, nous nous intéresserons seulement aux foncteurs covariants.

Ensuite, une catégorie possède une loi de composition et chaque objet doit
posséder une flèche identité; il serait probablement intéressant qu'une
transformation entre catégories préserve ces propriétés d'une certaine manière.
On demandera donc que la loi de composition soit respectée, c'est-à-dire que
pour des flèches $f : A \to B$ et $g : B \to C$ de $\C$,
\[
    F(g \circ_\C f) = F(g) \circ_\D F(f)
\]

Il est possible de visualiser cette équation sous la forme d'un diagramme.
En effet, dire que cette équation est satisfaite est équivalent à dire que
le diagramme suivant commute:
\[
\begin{tikzcd}
    A \arrow[r, red, "f" black] \arrow[d, blue, "F" black] & B \arrow[r, red, "g" black] & C \arrow[d, red, "F" black] \\
    F(A) \arrow[r, blue, "F(f)" black] & F(B) \arrow[r, blue, "F(g)" black] & F(C)
\end{tikzcd}
\]

En d'autres mots, on souhaite que les chemins bleus et rouges donnent
exactement la même réponse. Ensuite, au même titre qu'on souhaite que la loi
de composition soit respectée, on demandera que les flèches identité soient
préservées par un foncteur, c'est-à-dire que pour un objet $A$ de $\C$,
\[
    F(id_A) = id_{F(A)}
\]

Ceci nous mène à la définition formelle suivante:

\begin{définition}[Foncteur]
    Un foncteur $F : \C \to \D$ d'une catégorie $\C$ vers une catégorie $\D$
    est une application qui
    \begin{enumerate}
        \item à chaque objet $X \in \ob(\C)$ associe un objet $F(X) \in \ob(\D)$
        \item à chaque flèche $f : A \to B \in \hom(\C)$ associe une flèche
              $F(f) : F(A) \to F(B) \in \hom(\D)$
        \item respecte la loi de composition, c'est-à-dire que pour toutes
              flèches $f : A \to B$ et $g : B \to C \in \hom(\C)$,
              $F(g \circ_\C f) = F(g) \circ_\D F(f)$
        \item préserve les identités, c'est-à-dire que pour tout objet
              $X \in \ob(\C)$, $F(\id_X) = \id_{F(X)}$
    \end{enumerate}
\end{définition}

Lorsque le domaine et le codomaine d'un foncteur sont la même catégorie, i.e.
lorsqu'on a un foncteur du type $F : \C \to \C$, on dira parfois que $F$ est
un \textit{endofoncteur}.


\subsection{Les foncteurs dans Hana}
Dans le cadre de la bibliothèque Hana, on s'intéressera principalement aux
endofoncteurs sur la catégorie Hana, c'est-à-dire aux foncteurs du type
$Hana \to Hana$. D'abord, rappelons-nous que les objets de la catégorie
Hana sont des types généralisés. Un foncteur $F$ devra donc envoyer un type
généralisé \cc{T} vers un autre type généralisé $F(\cc{T})$. Ensuite, les
flèches de Hana sont des fonctions dont le domaine et le codomaine sont des
types généralisés. Ainsi, $F$ devra envoyer une telle fonction
$\cc{f} : \cc{T} \to \cc{U}$ vers une fonction $F(\cc{f}) : F(\cc{T}) \to F(\cc{U})$.
Dans Hana, un foncteur se matérialise comme un type généralisé paramétré muni
d'une fonction appelée \cc{transform} ayant la pseudo-signature suivante:
\[
    \cc{transform} : \cc{F(T)} \times (\cc{T} \to \cc{U}) \to \cc{F(U)}
\]

Comme on pouvait s'en douter, étant donné un type généralisé \cc{T}, \cc{F}
envoie \cc{T} sur un nouveau type généralisé \cc{F(T)}. Cependant, le rôle de
\cc{transform} demande un peu plus de justification. On notera que l'ordre des
paramètres de \cc{transform} est fâcheux pour l'explication qui suit. C'est
tout de même cet ordre qui est utilisé dans Hana parce qu'il est avantageux
lorsqu'on veut passer une fonction anonyme C++ à \cc{transform}. Supposons que
l'on ait une fonction $\cc{f} : \cc{T} \to \cc{U}$. On peut appliquer partiellement
\cc{f} à \cc{transform} pour obtenir la fonction
\[
    \cc{transform(-, f)} : \cc{F(T)} \to \cc{F(U)}
\]

On peut remarquer que \cc{transform(-, f)} possède la signature requise
pour être l'image d'une flèche de Hana par le foncteur \cc{F}. Dans la catégorie
Hana, un foncteur \cc{F} envoie donc un type généralisé \cc{T} sur un nouveau
type généralisé \cc{U}, et une fonction $\cc{f} : \cc{T} \to \cc{U}$ sur une
nouvelle fonction $\cc{transform(-, f)} : \cc{F(T)} \to \cc{F(U)}$. Bien entendu,
pour qu'il s'agisse bien d'un foncteur, certaines propriétés doivent être
respectées. D'abord, le foncteur doit respecter la loi de composition de Hana,
i.e. pour toutes fonctions $\cc{f} : \cc{A} \to \cc{B}$ et $\cc{g} : \cc{B} \to \cc{C}$,
on doit avoir que
\begin{cpp}
    compose(transform(-, g), f) == compose(transform(-, g), transform(-, f))
\end{cpp}

ce qui est équivalent à dire que pour tout \cc{xs} de type généralisé \cc{F(T)},
\begin{cpp}
    transform(xs, compose(g, f)) == transform(transform(xs, f), g)
\end{cpp}

Ensuite, le foncteur doit aussi respecter l'identité, c'est à dire que
$F(\id_{\cc{T}}) = \id_{\cc{F(T)}}$. Dans Hana, puisque l'identité est une
fonction générique qui fonctionne sur tous les types généralisés, ceci se
traduit simplement en disant que pour tout \cc{x} de type généralisé \cc{F(T)},
\begin{cpp}
    transform(x, id) == x
\end{cpp}

Voyons maintenant comment certaines structures fournies par Hana peuvent
être vues comme des foncteurs.


%%%%%%%%%%%%%%%%%%%%%%%%%%%%%%%%%%%%%%%%%%%%%%%%%%%%%%%%%%%%%%%%%%%%%%%%%%%%%%
\subsection{Le foncteur Maybe}
En programmation, on est souvent amené à utiliser des fonctions qui peuvent
échouer. Par exemple, lorsqu'on divise une quantité par une autre, il peut
arriver que le dénominateur soit nul, auquel cas la division ne peut avoir
lieu. Un autre exemple serait une fonction qui accède au n-ième élément
d'une séquence; il peut arriver que la séquence contienne moins de $n$
éléments, auquel cas la fonction ne peut pas retourner de résultat. Or,
en programmation, il nous faut une manière de gérer ces éventualités. Un
mécanisme classique pour y arriver est d'utiliser des exceptions. Cependant,
un des désavantages des exceptions est qu'elles ne sont généralement pas
encodées dans le type de la fonction qui peut les produire, et le fardeau
de bien les gérer appartient donc au programmeur. Un autre problème majeur
dans le contexte de la métaprogrammation est que les exceptions n'existent
pas au moment de la compilation, et ce mécanisme nous est donc complètement
inaccessible.

Une autre manière de gérer la possibilité d'un échec est d'introduire la
notion de valeur optionnelle. Essentiellement, une valeur optionnelle est
une valeur qui peut être là ou ne pas y être, accompagnée d'un mécanisme
pour savoir si la valeur est là ou non. Pour une valeur de type \cc{T},
on dénotera une valeur optionnelle de ce type par le type \cc{Maybe(T)}.
Une valeur optionnelle peut être créée de deux manières. D'abord, on peut
créer une valeur optionnelle (qui existe) avec \cc{just(x)}, où \cc{x}
est une valeur de type \cc{T}. Ensuite, on peut créer une valeur optionnelle
qui est vide avec \cc{nothing}. Par exemple, on pourrait écrire une fonction
qui effectue une division entière de la manière suivante (en écrivant \cc{Int}
au lieu de \cc{IntegralConstant(int)} pour alléger):
\begin{align*}
    \cc{div} : \cc{Int} \times \cc{Int} &\to \cc{Int} \\
                                   x, y &\mapsto x / y
\end{align*}

Cependant, on pourrait aussi écrire une fonction qui effectue la division de
manière sécuritaire en utilisant \icpp{Maybe}:
\begin{align*}
    \cc{safe_div} : \cc{Int} \times \cc{Int} &\to \cc{Maybe(Int)} \\
                    \cc{x}, \cc{y} &\mapsto \begin{cases}
                                        \cc{nothing} \text{ si \cc{y == int_<0>}} \\
                                        \cc{just(x / y)} \text{ sinon}
                                    \end{cases}
\end{align*}

Pour l'instant, ceci ne nous est pas très utile parce que nous n'avons pas
de façon d'extraire la valeur optionnelle lorsqu'elle existe. C'est le rôle
de la fonction \cc{from_just}, qui est définie par
\begin{align*}
    \cc{from_just} : \cc{Maybe(T)} &\to \cc{T} \\
                        \cc{just(x)} &\mapsto \cc{x} \\
                        \cc{nothing} &\mapsto \cc{erreur}
\end{align*}

Pour que ceci nous soit d'une quelconque utilité, il nous faut aussi une
manière de savoir quand est-ce qu'une valeur optionnelle est vide ou pleine:
\begin{align*}
    \cc{is_just} : \cc{Maybe(T)} &\to \cc{IntegralConstant(bool)} \\
                   \cc{just(x)} &\mapsto \cc{true_} \\
                   \cc{nothing} &\mapsto \cc{false_}
\end{align*}

On peut ensuite utiliser le résultat de \cc{safe_div} comme suit:
\begin{cpp}
    auto maybe_result = safe_div(x, y);
    if (is_just(maybe_result)) {
        auto result = from_just(maybe_result);
        ...
    }
\end{cpp}

On constate rapidement que le gain est plutôt faible par rapport à une
solution où on vérifierait explicitement que \cc{y != 0} et où l'on
n'utiliserait pas de \cc{Maybe}. En effet, il est quand même nécessaire
de vérifier si la valeur optionnelle est vide ou pleine avant de pouvoir
utiliser le résultat de la fonction (si résultat il y a). Ce qu'on aimerait
est une manière de seulement spécifier le contenu de la branche (par exemple
comme une fonction) et de soit appliquer la fonction au contenu du \cc{Maybe},
soit ne pas l'appliquer et avoir \cc{nothing}. Mathématiquement, si la
branche est représentée par une fonction $\cc{f} : \cc{A} \to \cc{B}$, on
aimerait (automatiquement) être capable de produire une fonction
\begin{align*}
    \cc{g} : \cc{Maybe(A)} &\to \cc{Maybe(B)}       \\
             \cc{just(x)} &\mapsto \cc{just(f(x))}  \\
             \cc{nothing} &\mapsto \cc{nothing}
\end{align*}

De cette manière, on pourrait simplement écrire
\begin{cpp}
    auto maybe_result = safe_div(x, y);
    g(maybe_result);
\end{cpp}

Comme on pouvait s'en douter, nous venons de mettre le doigt sur un foncteur.
En effet, étant donné un objet \cc{A} de la catégorie Hana, le foncteur
\cc{Maybe} envoie \cc{A} sur \cc{Maybe(A)}. Ensuite, étant donné une flèche
$\cc{f} : \cc{A} \to \cc{B}$, \cc{Maybe} envoie \cc{f} sur
\begin{align*}
    \cc{transform(-, f)} : \cc{Maybe(A)} &\to \cc{Maybe(B)}             \\
                           \cc{just(x)} &\mapsto \cc{just(f(x))}        \\
                           \cc{nothing} &\mapsto \cc{nothing}
\end{align*}

Afin de démontrer la fonctorialité de \cc{Maybe}, commençons par regarder
comment celui-ci est implémenté dans Hana. D'abord, on a \cc{just} et
\cc{nothing}, qui représentent des valeurs optionnelles.
\begin{cpp}
    template <typename T>
    struct _just { T value; };

    auto just = [](auto x) {
        return _just<decltype(x)>{x};
    };

    struct _nothing { };
    _nothing nothing{};
\end{cpp}

Ensuite, on a la fonction \cc{transform}, qui applique une fonction à une
valeur optionnelle et qui retourne un résultat optionnel. On notera
l'utilisation de l'\textit{overloading} dans l'implémentation:
\begin{cpp}
    template <typename T, typename F>
    auto transform(_just<T> x, F f) {
        return just(f(x.value));
    }

    template <typename F>
    auto transform(_nothing, F) {
        return nothing;
    }
\end{cpp}

Si un \cc{just(x)} est passé à \cc{transform}, on applique la fonction
\cc{f} à \cc{x} et on retourne le résultat comme un \cc{Maybe}. Sinon,
on retourne simplement \cc{nothing}, puisqu'il n'y a pas de valeur à
laquelle on pourrait appliquer \cc{f}. Montrons maintenant que \cc{Maybe}
est un endofoncteur sur Hana. D'abord, on doit montrer que \cc{transform}er
une valeur optionnelle avec la fonction identité ne change pas cette valeur
optionnelle. Soit donc $\cc{x} \in \cc{Maybe(A)}$ une valeur optionnelle de
type \cc{A}. Si \cc{x} est de la forme \cc{just(v)}, alors
\begin{cpp}
    transform(x, id) == transform(just(v), id)
                     == just(id(v))
                     == just(v)
                     == x
\end{cpp}

Si, au contraire, \cc{x} est un \cc{nothing}, alors
\begin{cpp}
    transform(x, id) == transform(nothing, id)
                     == nothing
\end{cpp}

ce qui montre que \cc{transform(-, id) == id}. Ensuite, il faut montrer
que la composition des fonctions est respectée par \cc{transform}. Soit
$\cc{x} \in \cc{Maybe(A)}$ une valeur optionnelle et $\cc{f} : \cc{A} \to \cc{B}$
et $\cc{g} : \cc{B} \to \cc{C}$ des fonctions quelconques. Si \cc{x} est de la
forme \cc{just(v)}, alors
\begin{cpp}
    transform(x, compose(g, f)) == transform(just(v), compose(g, f))
                                == just(compose(g, f)(v))
                                == just(g(f(v)))
\end{cpp}

d'un autre côté, on a que
\begin{cpp}
    transform(transform(x, f), g) == transform(transform(just(v), f), g)
                                  == transform(just(f(v)), g)
                                  == just(g(f(v)))
\end{cpp}

ce qui montre l'égalité recherchée. Ainsi, \cc{Maybe} est bien un foncteur
sur la catégorie Hana. De plus, avec cette définition de \cc{transform}, on
peut maintenant écrire
\begin{cpp}
    auto increment = [](auto x) { return x + int_<1>; };

    transform(safe_div(int_<4>, int_<2>), increment); // just(int_<3>)

    transform(safe_div(int_<4>, int_<0>), increment); // nothing
\end{cpp}

ce qui est un gain d'expressivité par rapport à notre solution initiale.
Cependant, qu'arrive-t-il si nous désirons composer plusieurs fonctions
qui peuvent potentiellement échouer? En d'autres mots, nous sommes
présentemment capable de composer des fonctions de la forme
$\cc{X} \to \cc{Maybe(Y)}$ et $\cc{Y} \to \cc{Z}$; la question
est de savoir comment nous pourrions composer des fonctions de la forme
$\cc{X} \to \cc{Maybe(Y)}$ et $\cc{Y} \to \cc{Maybe(Z)}$. C'est en effet
possible, mais il nous faudra attendre la section sur les monades pour y
arriver.


%%%%%%%%%%%%%%%%%%%%%%%%%%%%%%%%%%%%%%%%%%%%%%%%%%%%%%%%%%%%%%%%%%%%%%%%%%%%%%
\subsection{Le foncteur Tuple}
Jusqu'ici, nous avons vu comment faire des calculs arithmétiques et des
transformations sur des types au moment de la compilation. Nous avons aussi
vu comment il est possible de représenter une valeur optionnelle à l'aide d'un
foncteur, ce qui est particulièrement utile pour représenter le résultat
d'une fonction qui peut échouer. Nous allons maintenant voir comment il est
possible de manipuler des séquences d'objets, et comment la notion de foncteur
peut nous aider à formaliser ceci.

Étant donné une séquence d'objets $x_1, \hdots, x_n$ d'un ensemble $X$ et une
fonction $f : X \to Y$, il est parfois utile d'appliquer $f$ à chaque élément
de la séquence pour obtenir une nouvelle séquence $f(x_1), \hdots, f(x_n)$.
Ce \textit{pattern} étant omniprésent en programmation, on aimerait le
formaliser dans Hana et le généraliser, si possible. Premièrement, il nous
faut une séquence capable de contenir des objets d'un type généralisé \cc{X}
quelconque. On notera que puisque des objets d'un même type généralisé n'ont
pas forcément le même type C++, on ne peut pas utiliser une séquence
conventionnelle comme \icpp{std::vector}. On devra en effet utiliser une
séquence hétérogène capable de contenir des objets ayant différents types
C++. La véritable construction dans Hana est assez complexe pour des raisons
techniques, mais l'idée est essentiellement la suivante:
\begin{cpp}
    template <int i, typename T>
    struct _element { T value; };

    template <typename T1, ..., typename Tn>
    struct _tuple : _element<0, T1>, ..., _element<n-1, Tn> {
        // ... constructeur omis pour la simplicité
    };

    template <typename ...T>
    _tuple<T...> make_tuple(T ...t) {
        return {t...};
    }
\end{cpp}

On utilise le type \cc{_element<i, T>} pour stocker l'objet de type \cc{T}
situé à l'index \cc{i} dans la séquence. On se donne aussi un peu de sucre
syntaxique (\cc{make_tuple}), qui nous permettra de construire des
\cc{_tuple} sans devoir spécifier le type de chacun des éléments.
La raison pour laquelle on hérite de tous les \cc{_element}s dans
\cc{_tuple} et que cela nous permet d'accéder à chaque élément individuel
grâce à l'astuce suivante:
\begin{cpp}
    template <int i, typename T>
    T get(_element<i, T> e) {
        return e.value;
    }

    char y = get<3>(make_tuple('x', 'y', 'z'));
\end{cpp}

Puisque \cc{_tuple} hérite de tous les \cc{_element}s, il existe une
conversion de \cc{_tuple} vers n'importe laquelle de ses classes de base.
Ensuite, puisqu'on fixe le \cc{i} explicitement en appelant \cc{get<i>},
le compilateur constate que la seule signature possible pour \cc{get} est
\cc{char get(_element<3, char>)}. Il suffit ensuite de retourner la valeur
associée à cet élément. En bref, on a donc que
\cc{get<i>(make_tuple(x1, ..., xn)) == xi}.

Étant donné une séquence d'objets $\cc{x1}, \hdots, \cc{xn}$ de type généralisé
\cc{X}, on dénotera par \cc{Tuple(X)} le type généralisé de \cc{make_tuple(x1, ..., xn)}.
De cette manière, on considérera \cc{Tuple} comme un type généralisé paramétré,
où le paramètre représente le type généralisé des éléments que le \cc{Tuple}
contient. On notera en passant qu'il serait possible de stocker des éléments
ayant des types généralisés différents au sein d'un même \cc{_tuple}, puisque
\cc{_tuple} ne pose aucune restriction sur le type de ces éléments. Cependant,
on restreindra notre étude aux \cc{_tuple}s qui contiennent tous des éléments
d'un même type généralisé, puisque c'est ceux-là qui nous donnent les propriétés
les plus intéressantes.

Nous sommes maintenant en mesure de formuler concrètement le problème initial
qui était d'appliquer une fonction à chaque élément d'une séquence. Étant
donné une fonction $\cc{f} : \cc{X} \to \cc{Y}$ et une séquence \cc{make_tuple(x1, ..., xn)}
de type généralisé \cc{Tuple(X)}, on aimerait obtenir une séquence
\cc{make_tuple(f(x1), ..., f(xn))} (de type généralisé \cc{Tuple(Y)}).
Voici comment on pourrait faire ceci:
\begin{cpp}
    template <typename T1, ..., typename Tn, typename F>
    auto transform(_tuple<T1, ..., Tn> ts, F f) {
        return make_tuple(f(get<0>(ts)), ..., f(get<n-1>(ts)));
    }
\end{cpp}

Étant donné une séquence d'objets sous la forme d'un \cc{_tuple}, on
applique simplement \cc{f} à chaque élément (extrait avec \cc{get}),
puis on retourne une nouvelle séquence qui contient le résultat. À ce
point-ci, le foncteur caché sous le tapis devrait commencer à être évident.

En effet, en tant que foncteur, \cc{Tuple} associe à tout objet \cc{X} de $Hana$
un nouvel objet \cc{Tuple(X)} représentant une séquence d'éléments de type \cc{X},
et à toute flèche $\cc{f} : \cc{X} \to \cc{Y}$ une nouvelle flèche $\cc{transform(-, f)} :
\cc{Tuple(X)} \to \cc{Tuple(Y)}$, qui consiste à appliquer \cc{f} à chaque
élément d'une telle séquence.

Démontrons maintenant que \cc{Tuple} est bien un foncteur. D'abord, étant
donné un objet quelconque \cc{xs} de type généralisé \cc{Tuple(X)},
appliquer \cc{transform} à \cc{xs} et la fonction identité sur \cc{X}
ne change pas \cc{xs} du tout. En effet, en supposant que \cc{xs}
contienne $n$ éléments, on a alors
\begin{cpp}
    transform(xs, id) == make_tuple(id(get<0>(xs)), ..., id(get<n-1>(xs)))
                      == make_tuple(get<0>(xs), ..., get<n-1>(xs))
                      == xs
\end{cpp}

Ainsi, on a bien que \cc{transform(-, id) == id}, et \cc{Tuple} respecte
donc les flèches identités. Il reste à montrer que \cc{Tuple} respecte la
loi de composition, c'est-à-dire que pour toutes fonctions
$\cc{f} : \cc{X} \to \cc{Y}$ et $\cc{g} : \cc{Y} \to \cc{Z}$, on a
\begin{cpp}
    transform(transform(xs, f), g) == transform(xs, compose(g, f))
\end{cpp}

Or, on trouve facilement que
\begin{cpp}
    transform(transform(xs, f), g)
        == transform(make_tuple(f(get<0>(xs)), ..., f(get<n-1>(xs))), g)
        == make_tuple(g(f(get<0>(xs))), ..., g(f(get<n-1>(xs))))
        == make_tuple(compose(g, f)(get<0>(xs)), ..., compose(g, f)(get<n-1>(xs)))
        == transform(xs, compose(g, f))
\end{cpp}

ce qui montre que la loi de composition est bien respectée. Le fait d'exiger
que \cc{Tuple} soit un foncteur est bien plus qu'un simple caprice de
mathématicien. En effet, exiger que la composition soit respectée par
\cc{transform} permet de faire des optimisations pour éliminer des
structures temporaires inutiles. Par exemple, lorsque l'on applique
\begin{cpp}
    transform(transform(xs, f), g)
\end{cpp}

alors un \cc{_tuple} intermédiaire qui contient \cc{transform(xs, f)} doit
être créé, puis envoyé aux deuxième \cc{transform}. Si les objets retournés
par \cc{f} sont coûteux à copier, ceci peut représenter un problème. Or, la
loi des foncteurs nous dit qu'on peut réécrire cette ligne comme
\cc{transform(xs, compose(g, f))}, qui ne crée pas de structure
temporaire inutile.
