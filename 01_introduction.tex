\section{Introduction}
Certains langages de programmation utilisent des concepts tirés des
mathématiques pour faciliter le raisonnement à propos des programmes.
C'est notamment le cas de certains langages fonctionnels, qui, avec
Haskell en tête, utilisent des notions de théorie des catégories pour
définir des interfaces génériques encapsulant des phénomènes omniprésents
en programmation comme le non-déterminisme et la propagation d'erreurs.
Il en résulte des programmes qui sont généralement plus simples et surtout
plus souvent corrects.

À l'opposé du spectre des langages se trouve le C++, langage typiquement
associé à la programmation de bas niveau et peu reconnu pour ses qualités
d'expressivité, encore moins d'esthétisme mathématique. De plus, le coeur
du système de typage du C++ renferme une bête indomptée qui fut découverte
Turing complète par accident peu après sa création. Il s'agit du système des
\textit{templates}, dont les créateurs n'avaient initialement pas soupçonné
la puissance. Ce système forme un langage purement fonctionnel interprété par
le compilateur et qui a donné naissance à la métaprogrammation statique, outil
devenu omniprésent dans le C++ moderne de par ce qu'il permet d'accomplir.

L'utilisation des \textit{templates} comme modèle de calcul étant désagréable
au mieux, certaines bibliothèques comme la MPL \cite{mpl} et Fusion \cite{fusion}
sont apparues il y a quelques années, dans le but de rendre la métaprogrammation
plus accessible. Cependant, le langage C++ ayant beaucoup évolué dans les récentes
années, le domaine du possible s'est vu exploser et plusieurs bibliothèques de
métaprogrammation ont vu le jour, avec différents niveaux de succès.

Parmi ces bibliothèques se trouve Hana \cite{hana}, une bibliothèque qui
fusionne la programmation statique et la programmation dynamique en fournissant
une syntaxe unifiée pour exprimer les calculs, ce qui n'avait jamais été fait
auparavant. Les outils fournis par cette bibliothèque sont spécifiés dans
un langage mathématique semi-formel dont les idées sont tirées du Haskell,
de la théorie des catégories et parfois carrément nouvelles. Ce texte se
veut à la fois une introduction à la métaprogrammation moderne en C++ et
une formalisation de celle-ci à l'aide de la théorie des catégories, en
utilisant la bibliothèque Hana comme exemple concret de cette interaction.
