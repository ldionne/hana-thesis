\section{Les monades}

\todo{Introduction avec un peu d'intuition; on s'en va tranquillement vers
la définition formelle.}

\begin{définition}[Monade]
    Soit $\C$ une catégorie et $I_\C : \C \to \C$ le foncteur identité sur
    $\C$. Une monade est un endofoncteur $F : \C \to \C$ accompagné de deux
    transformation naturelles $\eta : I_\C \to F$ et $\mu : F \circ F \to F$
    telles que pour tout objet $X \in \ob(\C)$ et toute flèche
    $f : X \to Y \in \hom(\C)$,
    \begin{enumerate}
        \item $\mu_X \circ F(\mu_X) = \mu_X \circ \mu_{F(X)}$
        \item $\mu_X \circ F(\eta_X) = \mu_X \circ \eta_{F(X)} = id_{F(X)}$
        \item $\eta \circ f = F(f) \circ \eta$
        \item $\mu \circ F(F(f)) = F(f) \circ \mu$
    \end{enumerate}
\end{définition}

Pour déchiffrer cette définition un peu opaque, commençons par donner un nom
plus descriptif aux transformations naturelles $\eta$ et $\mu$. Puisque $\eta$
associe à tout objet $X \in \ob(\C)$ une flèche $\eta_X : I_\C(X) = X \to F(X)$,
on l'appellera $lift$. L'intuition est que $lift$ sert à prendre un objet de
$\C$ pour le ``monter'' dans le foncteur $F$. Ensuite, puisque $\mu$ associe
à tout objet $X \in \ob(\C)$ une flèche $\mu_X : F(F(X)) \to F(X)$, on
l'appellera $flatten$. Ici, l'intuition est que $flatten$ prend un objet
qui est appliqué deux fois à un foncteur et qu'il élimine, ou écrase, un
niveau de fonctorialité.

\todo{Je dois présenter un exemple de Monade dans Hana, puis déchiffrer les
lois à l'aide de cet exemple. Sinon, c'est pas intéressant.}

En reformulant la définition d'une monade de cette manière, on obtient que
les deux transformations naturelles doivent respecter les équations suivantes:
\begin{enumerate}
    \item $flatten_X \circ F(flatten_X) = flatten_X \circ flatten_{F(X)}$\\
    Du côté gauche, on a que $flatten_X$ est une flèche de type $F(F(X)) \to F(X)$,
    et donc $F(flatten_X)$ est une flèche de type $F(F(F(X))) \to F(F(X))$. Ainsi,
    la composition $flatten_X \circ F(flatten_X)$ fait du sens et nous donne une
    flèche de type $F(F(F(X))) \to F(X)$. Notons que cette flèche fait bien
    partie de la catégorie $\C$, puisque $F$ est un endofoncteur. Du côté
    droit, on a $flatten_X \circ flatten_{F(X)}$, ce qui nous donne une flèche
    de type $F(F(F(X))) \to F(X)$. L'équation fait donc bien du sens.
    Intuitivement, cette équation représente le fait que \todo{intuition}

    \item $flatten_X \circ F(lift_X) = flatten_X \circ lift_{F(X)} = id_{F(X)}$\\
    Du côté gauche, on a $flatten_X : F(F(X)) \to F(X)$ et
    $F(lift_{X}) : F(X) \to F(F(X))$. La composition donne donc une flèche
    $flatten_X \circ F(lift_X) : F(X) \to F(X)$.

    Du côté droit, on a $flatten_X : F(F(X)) \to F(X)$ et
    $lift_{F(X)} : F(X) \to F(F(X))$. La composition donne donc aussi une
    flèche $F(X) \to F(X)$.
\end{enumerate}
