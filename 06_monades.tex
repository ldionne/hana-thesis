\section{Les monades}
Une monade est un endofoncteur avec davantage de structure. Cette structure
additionnelle nous permet de représenter plusieurs phénomènes courants en
programmation qui, avant l'arrivée des monades, étaient considérés comme
complètement déconnectés des mathématiques.

\begin{définition}[Monade]
    Soit $\C$ une catégorie et $I_\C : \C \to \C$ le foncteur identité sur
    $\C$. Une monade est un endofoncteur $F : \C \to \C$ accompagné de deux
    transformation naturelles $\eta : I_\C \to F$ et $\mu : F \circ F \to F$
    telles que pour tout objet $X \in \ob(\C)$ et toute flèche
    $f : X \to Y \in \hom(\C)$,
    \begin{enumerate}
        \item $\mu_X \circ F(\mu_X) = \mu_X \circ \mu_{F(X)}$
        \item $\mu_X \circ F(\eta_X) = \mu_X \circ \eta_{F(X)} = id_{F(X)}$
        \item $\eta_Y \circ f = F(f) \circ \eta_X$
        \item $\mu_Y \circ F(F(f)) = F(f) \circ \mu_X$
    \end{enumerate}
\end{définition}

\todo{Diagramme commutatif monades?}

%%%%%%%%%%%%%%%%%%%%%%%%%%%%%%%%%%%%%%%%%%%%%%%%%%%%%%%%%%%%%%%%%%%%%%%%%%%%%%
\subsection{Les monades dans Hana}
Pour déchiffrer cette définition un peu opaque, regardons maintenant comment
une monade est représentée dans Hana. Premièrement, on donne un nom plus
descriptif aux transformations naturelles $\eta$ et $\mu$. Puisque $\eta$
associe à tout objet $X \in \ob(\C)$ une flèche $\eta_X : I_\C(X) = X \to F(X)$,
on l'appellera \cc{lift}. Plus précisément, étant donné un type généralisé
paramétré \cc{F} et un type généralisé \cc{X}, $\cc{lift<X>} : \cc{X} \to \cc{F(X)}$
prend un objet de \cc{X} et ``l'injecte'' ou le ``monte'' dans le foncteur \cc{F}.

Ensuite, puisque $\mu$ associe à tout objet $X \in \ob(\C)$ une flèche
$\mu_X : F(F(X)) \to F(X)$, on l'appellera \cc{flatten}. Ici, l'intuition
est que $\cc{flatten<X>} : \cc{F(F(X))} \to \cc{F(X)}$ prend un objet
enveloppé dans deux niveaux de fonctorialité et élimine, ou écrase, un niveau
de fonctorialité. Pour alléger la notation, on notera parfois simplement
\cc{flatten} et \cc{lift} au lieu de \cc{flatten<...>} et \cc{lift<...>}.
En reformulant la définition d'une monade de cette manière, on obtient que
les deux transformations naturelles doivent respecter les équations suivantes.
D'abord, on doit avoir que
\begin{cpp}
    compose(flatten<X>, transform(-, flatten<X>))
        == compose(flatten<X>, flatten<F(X)>)
\end{cpp}

ce qui est équivalent à dire que pour tout objet $\cc{xs} \in \cc{F(F(F(X)))}$,
\begin{cpp}
    flatten(transform(xs, flatten)) == flatten(flatten(xs))
\end{cpp}

Considérant \cc{xs} comme une structure à ``3 niveaux'', cette loi dit donc
qu'écraser les niveaux intérieurs puis ensuite les niveaux extérieurs est
équivalent à écraser les niveaux extérieurs puis ensuite les niveaux
intérieurs. Quant à la deuxième loi, elle peut s'écrire comme
\begin{cpp}
    compose(flatten<X>, transform(-, lift<X>))
        == compose(flatten<X>, lift<F(X)>)
        == id
\end{cpp}

ce qui est équivalent à dire que pour tout objet $\cc{xs} \in \cc{F(X)}$,
\begin{cpp}
    flatten(transform(xs, lift)) == flatten(lift(xs)) == xs
\end{cpp}

D'abord, à droite on a \cc{flatten(lift(xs)) == xs}, ce qui veut simplement
dire que monter une valeur dans un foncteur puis écraser ce niveau de
fonctorialité devrait être équivalent à ne rien faire du tout. À gauche, on
a que \cc{flatten(transform(xs, lift)) == xs}, ce qui veut dire que monter
les valeurs dans un foncteur puis ensuite écraser ce nouveau niveau de
fonctorialité devrait être équivalent à ne rien faire. Cette loi permet
de s'assurer que \cc{flatten} et \cc{lift} ne cachent pas d'effets arbitraires.
Pour la suite, supposons que \cc{f} est une fonction $\cc{X} \to \cc{Y}$. La
troisième loi peut s'écrire comme
\begin{cpp}
    compose(lift<Y>, f) == compose(transform(-, f), lift<X>)
\end{cpp}

De manière équivalente, on peut dire que pour tout objet $\cc{x} \in \cc{X}$,
\begin{cpp}
    lift(f(x)) == transform(lift(x), f)
\end{cpp}

Ceci représente le fait qu'appliquer une fonction à une valeur simple montée
avec \cc{lift} à l'intérieur du foncteur est la même chose que d'effectuer
la transformation à l'extérieur du foncteur, puis de monter le résultat dans
le foncteur. Finalement, étant donné un objet $\cc{xs} \in \cc{F(F(X))}$, on
peut écrire la quatrième loi comme
\begin{cpp}
    flatten(transform(xs, transform(-, f))) == transform(flatten(xs), f)
\end{cpp}

Il est plus difficile de développer une intuition pour cette loi, mais les
exemples de monades qui suivent la rendront probablement plus claire. Avant
de donner des exemples utiles de l'utilisation des monades dans Hana, il nous
faut introduire une fonction nommée \cc{bind}, qui sert à enchaîner des
fonctions dont le retour est monadique. Plus précisément, \cc{bind} est
définie comme
\begin{align*}
    \cc{bind} : \; &\cc{F(X)} \times (\cc{X} \to \cc{F(Y)}) \to \cc{F(Y)} \\
                &\cc{xs} \times \cc{f} \mapsto \cc{flatten(transform(xs, f))}
\end{align*}

Étant donné une valeur monadique $\cc{xs} \in \cc{F(X)}$ et une fonction à
retour monadique $\cc{f} : \cc{X} \to \cc{F(Y)}$, \cc{bind} applique la
fonction \cc{f} à l'intérieur de \cc{xs}, ce qui nous donne un objet de
type \cc{F(F(Y))}. Ensuite, \cc{bind} applique \cc{flatten} pour obtenir
un résultat final de type \cc{F(Y)}. Une manière possible de voir \cc{bind}
est comme l'application d'une fonction à une valeur, mais une application
à laquelle on aurait ajouté des effets représentés par la monade \cc{F}.
Pour alléger la notation, Hana définit l'opérateur \cc{operator|} comme
étant équivalent à \cc{bind}: \cc{xs | f == bind(xs, f)}. Nous utiliserons
parfois cette notation par la suite. Nous sommes maintenant prêts pour voir
des exemples concrets de Monades qui sont utilisées dans Hana.


%%%%%%%%%%%%%%%%%%%%%%%%%%%%%%%%%%%%%%%%%%%%%%%%%%%%%%%%%%%%%%%%%%%%%%%%%%%%%%
\subsection{La monade Maybe}
Nous avons vu précédemment que \cc{Maybe} était un foncteur. En fait,
\cc{Maybe} est plus qu'un foncteur; c'est aussi une monade. D'abord,
l'opération \cc{lift} devrait avoir une signature $\cc{X} \to \cc{Maybe(X)}$.
La définition la plus naturelle pour une fonction ayant cette signature est
$\cc{lift} = \cc{just}$.

Ainsi, \cc{lift} prend une valeur normale et la ``monte'' dans une valeur
optionnelle. Ensuite, la fonction \cc{flatten} devrait avoir une signature
$\cc{Maybe(Maybe(X))} \to \cc{Maybe(X)}$. Encore une fois, la définition la
plus naturelle est la bonne:
\begin{align*}
    \cc{flatten} : \cc{Maybe(Maybe(X))} &\to \cc{Maybe(X)} \\
    \cc{just(just(x))} &\mapsto \cc{just(x)} \\
    \cc{just(nothing)} &\mapsto \cc{nothing} \\
    \cc{nothing} &\mapsto \cc{nothing}
\end{align*}

Montrons maintenant qu'il s'agit bien d'une monade. Pour montrer que la
première loi tient, considérons $\cc{mmm} \in \cc{Maybe(Maybe(Maybe(X)))}$.
Il y a quatre cas possibles.
\begin{enumerate}
    \item Si $\cc{mmm} = \cc{nothing}$, alors
    \begin{cpp}
        flatten(transform(nothing, flatten))
            == flatten(nothing)
            == nothing
            == flatten(flatten(nothing))
    \end{cpp}

    \item Si $\cc{mmm} = \cc{just(nothing)}$, alors
    \begin{cpp}
        flatten(transform(just(nothing), flatten))
            == flatten(just(flatten(nothing)))
            == flatten(just(nothing))
            == nothing
            == flatten(flatten(just(nothing)))
    \end{cpp}

    \item Si $\cc{mmm} = \cc{just(just(nothing))}$, alors
    \begin{cpp}
        flatten(transform(just(just(nothing)), flatten))
            == flatten(just(flatten(just(nothing))))
            == flatten(just(nothing))
            == nothing
            == flatten(flatten(just(just(nothing))))
    \end{cpp}

    \item Si au contraire on a $\cc{mmm} = \cc{just(just(just(x)))}$, alors
    \begin{cpp}
        flatten(transform(just(just(just(x))), flatten))
            == flatten(just(flatten(just(just(x)))))
            == flatten(just(just(x)))
            == flatten(flatten(just(just(just(x)))))
    \end{cpp}
\end{enumerate}

Ceci montre que la première loi est bien respectée. Pour la deuxième loi,
considérons maintenant $\cc{m} \in \cc{Maybe(X)}$. Il y a deux cas possibles.
\begin{enumerate}
    \item Si $\cc{m} = \cc{nothing}$, alors
    \begin{cpp}
        flatten(transform(nothing, lift)) == flatten(nothing)
                                          == nothing
                                          == flatten(just(nothing))
                                          == flatten(lift(nothing))
    \end{cpp}

    \item Si $\cc{m} = \cc{just(x)}$, alors
    \begin{cpp}
        flatten(transform(just(x), lift)) == flatten(just(lift(x)))
                                          == flatten(just(just(x)))
                                          == flatten(lift(just(x)))
                                          == just(x)
    \end{cpp}
\end{enumerate}

La troisième loi est facile à montrer. Soit $\cc{x} \in \cc{X}$ et
$\cc{f} : \cc{X} \to \cc{Y}$. Alors
\begin{cpp}
    lift(f(x)) == just(f(x))
               == transform(just(x), f)
               == transform(lift(x), f)
\end{cpp}

Finalement, pour un $\cc{mm} \in \cc{F(F(X))}$, la quatrième loi nous donne
\begin{enumerate}
    \item Si $\cc{mm} = \cc{nothing}$, alors
    \begin{cpp}
        flatten(transform(nothing, transform(-, f)))
            == flatten(nothing)
            == nothing
            == transform(nothing, f)
            == transform(flatten(nothing), f)
    \end{cpp}

    \item Si $\cc{mm} = \cc{just(nothing)}$, alors
    \begin{cpp}
        flatten(transform(just(nothing), transform(-, f)))
            == flatten(just(transform(nothing, f)))
            == flatten(just(nothing))
            == nothing
            == transform(nothing, f)
            == transform(flatten(just(nothing)), f)
    \end{cpp}

    \item Si $\cc{mm} = \cc{just(just(x))}$, alors
    \begin{cpp}
        flatten(transform(just(just(x)), transform(-, f)))
            == flatten(just(transform(just(x), f)))
            == flatten(just(just(f(x))))
            == just(f(x))
            == transform(just(x), f)
            == transform(flatten(just(just(x))), f)
    \end{cpp}
\end{enumerate}

Ceci montre que \cc{Maybe} est bien une monade. Voici une manière de nous en
servir. D'abord, définissons la fonction \cc{sfinae}, qui nous permettra de
manipuler des expressions potentiellement invalides sans causer d'erreur de
compilation. Étant donné une fonction \cc{f}, \cc{sfinae(f)} est une fonction
qui applique \cc{f} à son argument si cet appel est valide, et qui retourne
\cc{nothing} sinon. Voici comment on peut l'implémenter:
\begin{cpp}
    template <typename F, typename ...X>
    constexpr auto sfinae_impl(int, F f, X ...x) -> decltype(just(f(x...))) {
        return just(f(x...));
    }

    template <typename F, typename ...X>
    constexpr auto sfinae_impl(long, F, X ...) {
        return nothing;
    }

    auto sfinae = [](auto f) {
        return [=](auto ...args) {
            return sfinae_impl(int{}, f, args...);
        };
    };
\end{cpp}

D'abord, \cc{sfinae(f)} est juste une fonction qui appelle \cc{sfinae_impl}
avec les arguments qu'on lui passe. On aurait aussi pu décider de passer les
arguments directement à \cc{sfinae}, i.e. d'écrire \cc{sfinae(f, args...)}
au lieu de \cc{sfinae(f)(args...)}. On décide de curryifier le premier argument
de \cc{sfinae} parce que ceci facilite généralement son utilisation. Le corps
de la technique se passe dans l'appel à \cc{sfinae_impl}.
D'abord, puisque \cc{sfinae_impl} est surchargée mais que \cc{sfinae} l'appelle
avec un argument de type \cc{int}, la première version sera préférée par le
compilateur (la deuxième version nécessite une conversion implicite de \cc{int}
vers \cc{long}). Cependant, si l'expression \cc{just(f(args...))} est invalide,
la première version de \cc{sfinae_impl} possédera alors une signature invalide,
et le compilateur devra plutôt utiliser la deuxième surcharge, sans toutefois
causer d'erreur de compilation parce que la première fonction ne pouvait pas
être choisie. Ce phénomène appelé \textit{SFINAE} (\textit{Substitution
Failure Is Not An Error}) est couramment utilisé dans les bibliothèques de
programmation générique pour tester la validité d'une expression. Donc, si
\cc{just(f(args...))} est une expression valide, alors la première fonction
sera choisie et le résultat sera \cc{just(f(args...))}. Sinon, la deuxième
fonction sera choisie et le résultat sera \cc{nothing}. Par exemple,
définissons un type \cc{Person} avec deux attributs ainsi que des
fonctions permettant d'accéder à ces attributs:
\begin{cpp}
    struct Person {
        std::string name;
        unsigned int age;
        // ...
    };

    auto name(Person p) { return p.name; }
    auto age(Person p) { return p.age; }
\end{cpp}

Comme on s'y attendrait, tenter d'appeler \cc{age} ou \cc{name} sur autre
chose qu'un objet de type \cc{Person} produirait une erreur de compilation,
puisque \cc{name} est définie seulement pour des objets de type \cc{Person}:
\begin{cpp}
    // error: no matching function for call to 'name'
    name(1);
\end{cpp}

Pour éviter une erreur de compilation et prendre en main la gestion de
l'invalidité de l'appel, nous pouvons utiliser \cc{sfinae}:
\begin{cpp}
    Person john{"John", 30};
    sfinae(name)(john); // just("John")
    sfinae(name)(1);    // nothing
\end{cpp}

Mais qu'arrive-t-il si nous souhaitons composer des fonctions qui peuvent
échouer? C'est là que la monadicité de \cc{Maybe} va nous venir en aide.
Considérons \cc{f} une fonction qui prend un pointeur vers une \cc{Person}
et qui renvoie la longueur de son nom. On voudrait de plus que \cc{f} nous
renvoie un \cc{nothing} plutôt que de faire échouer la compilation si
\textit{n'importe quelle étape} de la chaîne est invalide:
\begin{cpp}
    auto deref = [](auto x) -> decltype(*x) { return *x; };
    auto length = [](auto x) -> decltype(x.length()) { return x.length(); };

    auto f = [](auto x) {
        return sfinae(deref)(x) | sfinae(name)
                                | sfinae(length);
    };
\end{cpp}

On utilise \cc{bind} (sous la forme \cc{operator|}) pour créer une
composition monadique des fonctions \cc{sfinae(deref)}, \cc{sfinae(name)}
et \cc{sfinae(length)}. De cette manière,
\begin{cpp}
    // `nothing` car on tente de déréférencer un non-pointeur.
    f(john);

    // `nothing` car on tente d'accéder à l'attribut `name` dans un
    // objet de type `int`, qui n'en a pas.
    f(static_cast<int*>(0));

    // `nothing` car on tente d'appeler `.length()` sur un objet de
    // type `long`, ce qui est invalide.
    struct Foo { long name; };
    Foo foo;
    f(&foo);

    // `just(4u)`; tout va bien.
    f(&john);
\end{cpp}

Ainsi, la monade \cc{Maybe} nous permet de composer des fonctions qui peuvent
échouer. Si on ne s'était pas rendu compte qu'il s'agissait d'une monade, il
est peu probable que nous aurions trouvé un interface aussi composable pour
manipuler ces fonctions. De plus, le fait que \cc{Maybe} soit une monade
rend possible l'utilisation de plusieurs algorithmes génériques qui
fonctionnent pour n'importe quelle monade, par exemple les accumulations
monadiques (\textit{monadic folds}), qui ne sont pas traités ici. Les
algorithmes de cette famille peuvent être utilisés (avec \cc{Maybe}) pour
réduire une structure à l'aide d'une opération qui peut échouer à chaque
étape de réduction, ce qui mène à un gain d'expressivité et de généralité
significatif sur les alternatives existantes en métaprogrammation.


%%%%%%%%%%%%%%%%%%%%%%%%%%%%%%%%%%%%%%%%%%%%%%%%%%%%%%%%%%%%%%%%%%%%%%%%%%%%%%
\subsection{La monade Tuple}
Un autre foncteur que nous avons vu dans une section précédente est \cc{Tuple}.
Tout comme \cc{Maybe}, il s'agit non seulement d'un foncteur mais aussi d'une
monade. Afin d'alléger la notation, on dénotera par \cc{[x1, ..., xn]}
l'expression \cc{make_tuple(x1, ..., xn)}. Premièrement, on peut définir
\cc{lift} comme
\begin{align*}
    \cc{lift} : \cc{X} &\to \cc{Tuple(X)} \\
                \cc{x} &\mapsto \cc{[x]}  \\
\end{align*}

\cc{lift} prend simplement un élément et le met dans un tuple singleton.
Ensuite, on peut définir \cc{flatten} comme
\begin{align*}
    \cc{flatten} : \cc{Tuple(Tuple(X))} &\to \cc{Tuple(X)} \\
                   \cc{[[x1, ..., xn], ..., [y1, ..., ym]]} &\mapsto \cc{[x1, ..., xn, y1, ..., ym]}
\end{align*}

Montrons maintenant que \cc{Tuple} est bien une monade. D'abord, pour montrer
la première loi, considérons $\cc{xss} \in \cc{Tuple(Tuple(Tuple(X)))}$. Pour
simplifier la notation, on écrira \cc{xss = [[[x...]...]...]}, et \cc{x... ...}
représentera la concaténation des listes contenues dans \cc{[[x...]...]}.
On a alors
\begin{cpp}
    flatten(transform([[[x...]...]...], flatten))
        == flatten([flatten([[x...]...]])...])
        == flatten([[x... ...]...])
        == flatten(flatten([[[x...]...]...]))
\end{cpp}

Ensuite, pour la deuxième loi, on a bien que pour tout
$\cc{xs = [x...]} \in \cc{Tuple(X)}$,
\begin{cpp}
    flatten(transform([x...], lift)) == flatten([lift(x)...])
                                     == flatten([[x]...])
                                     == [x...]
                                     == flatten([[x...]])
                                     == flatten(lift([x...]))
\end{cpp}

La troisième loi est facile à voir. Étant donné un objet $\cc{x} \in \cc{X}$
et une fonction $\cc{f} : \cc{X} \to \cc{Y}$, on a bien
\begin{cpp}
    lift(f(x)) == [f(x)]
               == transform([x], f)
               == transform(lift(x), f)
\end{cpp}

Finalement, la quatrième et dernière loi est elle aussi respectée, puisque
pour un objet $\cc{xs = [[x...]...]} \in \cc{Tuple(Tuple(X))}$, on a
\begin{cpp}
    flatten(transform([[x...]...], transform(-, f)))
                        == flatten([transform([x...], f)...])
                        == flatten([[f(x)...]...])
                        == [f(x)... ...]
                        == transform([x... ...], f)
                        == transform(flatten([[x...]...]), f)
\end{cpp}

Ceci montre que \cc{Tuple} est bien une monade. Comment ceci peut-il nous
être utile? Il se trouve que \cc{Tuple} peut être utilisé pour composer des
fonctions ``non déterministes''. En effet, une fonction $\cc{f} : \cc{X} \to
\cc{Tuple(Y)}$ peut être vue comme une fonction non déterministe qui associe
plusieurs résultats possibles à chaque valeur d'entrée. Or, la fonction
\cc{bind} associée à \cc{Tuple} possède la signature $\cc{Tuple(X)} \times
(\cc{X} \to \cc{Tuple(Y)}) \to \cc{Tuple(Y)}$. Ainsi, elle permet de passer
le résultat d'une fonction ayant plusieurs valeurs possibles (un \cc{Tuple(X)})
à une fonction non déterministe acceptant un seul \cc{X}. En utilisant
\cc{bind} en chaîne, il est possible de créer un pipeline de transformations
non déterministes sans que cela ne soit trop encombrant. Par exemple,
supposons qu'on ait une fonction générique f:
\begin{cpp}
    template <typename T>
    auto f(T&& t) {
        // ...
    }
\end{cpp}

\cc{f} étant un \textit{template}, son corps n'est pas instancié par le
compilateur avant qu'on en fasse la requête pour un type \cc{T} explicite.
Ainsi, il est possible que la fonction soit invalide pour certains choix de
\cc{T} (pour lesquels on aimerait qu'elle soit valide), mais que cette erreur
ne soit attrapée par le compilateur que beaucoup plus tard, lorsqu'on appelle
la fonction avec un des types \cc{T} problématique. Ceci pose particulièrement
problème pour tester les bibliothèques génériques qui sont presque uniquement
constituées de \textit{templates}. Or, il se trouve que la monade \cc{Tuple}
peut nous aider à s'assurer que la fonction \cc{f} compile bien avec toutes
les variations possibles d'un type donné. D'abord, écrivons une fonction
$\cc{cv_qualifiers} : \cc{Type} \to \cc{Tuple(Type)}$ qui associe à un
\cc{Type} une liste de ses variations \textit{cv-qualifiées}:
\begin{cpp}
    template <typename T>
    auto cv_qualifiers(_type<T>) {
        return make_tuple(
            type<T>, type<T const>, type<T volatile>, type<T const volatile>
        );
    }
\end{cpp}

Ainsi, étant donné un type \cc{T}, \cc{cv_qualifiers} retourne un \cc{Tuple}
qui contient toutes les variations possibles de qualifications \cc{const} et
\cc{volatile}. De manière similaire, définissons une fonction $\cc{ref_qualifiers}
: \cc{Type} \to \cc{Tuple(Type)}$ qui associe à un type une liste des références
possibles à ce type.
\begin{cpp}
    template <typename T>
    auto ref_qualifiers(_type<T>) {
        return make_tuple(type<T&>, type<T&&>);
    }
\end{cpp}

Maintenant, étant donné un type \cc{T}, disons \cc{int}, il nous est possible
de générer toutes les combinaisons de qualificatifs \cc{const-volatile} et de
références en \cc{bind}ant le retour de \cc{cv_qualifiers} dans l'entrée de
\cc{ref_qualifiers}, ou vice-versa. Finalement, on peut utiliser la fonction
\cc{sfinae} présentée plus tôt pour s'assurer que la fonction \cc{f} compile
bien avec toutes les variations de types d'arguments qui nous intéressent:
\begin{cpp}
    // args == [int&, int&&, int const&, in const&&, int volatile&,
    //          int volatile&&, int const volatile&, int const volatile&&]
    auto args = bind(cv_qualifiers(type<int>), ref_qualifiers);

    transform(args, [](auto arg) {
        using T = typename decltype(arg)::type;
        using Valid = decltype(is_just(sfinae(f)(std::declval<T>())));
        static_assert(Valid{}, "");

        return 0; // on doit retourner quelque chose pour transform
    });
\end{cpp}

D'abord, on génère la séquence de tous les types d'arguments possible qu'on
souhaite peut-être envoyer à \cc{f}. Ensuite, on utilise \cc{transform} pour
appliquer une fonction anonyme à chacun de ces types. Étant donné un type
d'argument sous la forme d'un \cc{type<T>}, cette fonction anonyme extrait le
type \cc{T}, puis utilise \cc{sfinae} pour s'assurer que l'appel à la fonction
\cc{f} avec ce type d'argument est bien formé.
